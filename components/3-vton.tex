\section{Virtual try-on} \label{section:vton}
	Virtual try-On systems represent a cutting-edge fusion of technology and fashion retail, revolutionizing the way consumers interact with clothing online. These systems utilize artificial intelligence, computer vision, and augmented reality to simulate the experience of trying on clothing virtually. Users can see how garments fit, drape, and look on their own bodies without physically trying them on. From 2D image-based try-ons to advanced 3D models, these systems offer a range of immersive experiences.
	
	The goal of virtual try-on is not only to enhance user confidence and satisfaction, but also to reduce return rates, making it a pivotal tool in modern e-commerce. It showcases the potential of technology to bridge the gap between digital and physical retail, offering consumers an engaging and informative way to make fashion choices online.

	\subsection{Image-based (2D) virtual try-on}
		Image-based try-on systems take in images as input data and generate images as output data. The systems infer data like pose, warp, and occlusion of the target and then use that information to guide the generation of the post-try-on image.

	\subsection{Pose-guided human synthesis}
		These systems are also image-based, but they focus on transforming a human image from reference to target pose while preserving style but changing clothing.

	\subsection{Multi-pose guided virtual try-on}
		These systems are a step up from pose-guided systems; given a input image of a person, the target clothing, and a target pose, these attempt to generate the person in the target clothing in the target pose.

	\subsection{Video virtual try-on}
		Video virtual try-on systems fit target clothes onto a person in a video with spatio-temporal consistency. This is challenging because usual image-based try-on methods cause frame-to-frame inconsistencies when applied to video data.

	\subsection{3D virtual try-on}
		3D virtual try-on systems reconstruct 3D meshes of the person and clothing from the target images and then fit the clothing onto the person in attempts to generate a physically accurate 3D render.

	\subsection{Augmented reality try-on}
		Augmented reality virtual try-on is the eventual goal of all try-on systems, integrating computer-generated imagery with real-world views, enabling users to virtually try on clothing and accessories in real-time.
		
	\subsection{Commercial uses of virtual try-on}
		Companies are also using virtual try-on systems to provide enhanced experience to customers. Though many of them are limited to items like accessories and make-up, clothing try-ons are not far away.
